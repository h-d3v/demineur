\section{Classe Partie}

Le résultat d'une partie de démineur.

Le niveau de difficulté d'une partie est déterminé selon la taille de la grille et le nombre de mines comme suit :

\begin{enumerate}
\item Facile : 9x9, 10 mines
\item Moyen : 16x16, 40 mines
\item Difficile : 24x24, 99 mines
\end{enumerate}

\begin{itemize}
\item Propriétés : 
  \begin{enumerate}
  \item id : L'identifiant unique de la partie
    \begin{itemize}
    \item Type : int
    \item Accès : lecture seule
    \end{itemize}
  \item temps : Temps écoulé en secondes depuis le début de la partie ou jusqu'à ce qu'elle soit terminée.
    \begin{itemize}
    \item Type : 
    \item Accès : lecture seule
    \end{itemize}
  \item date : Date de début de la partie
    \begin{itemize}
    \item Type : LocalDateTime
    \item Accès : lecture seule
    \end{itemize}
  \item date : Date de fin de la partie
    \begin{itemize}
    \item Type : LocalDateTime
    \item Accès : complet
    \end{itemize}
  \item niveauDifficulté : Niveau de difficulté de la partie.
    \begin{itemize}
    \item Type : NiveauDifficulté
    \item Accès : lecture seule
    \end{itemize}

  \end{enumerate}

\item Constructeurs : 

  \begin{enumerate}
  \item Partie : Constructeur complet. Utile pour construire une partie en précisant toutes ses caractéristiques.
    \begin{itemize}
    \item Paramètres : 
      \begin{enumerate}
      \item unID (int) : identifiant unique de la partie.
      \item uneDateDébut (LocalDateTime) : La date et l'heure de début de la partie
      \item uneDateFin (LocalDateTime) : La date et l'heure de fin de la partie, null si le partie est en cours.
      \item niveauDifficulté (Niveaudifficulté) : Le niveau de difficulté de la partie.
      \end{enumerate}
    \item Assertions : 
      \begin{enumerate}
      \item uneDateFin est null ou après uneDateDébut.
      \end{enumerate}
    \end{itemize}
  \item Partie : Constructeur avec valeurs par défaut. Utile pour créer une toute nouvelle partie. La partie est créée avec la date et l'heure courante et un id=-1.
    \begin{itemize}
    \item Paramètres : 
      \begin{enumerate}
      \item niveauDifficulté (Niveaudifficulté) : Le niveau de difficulté de la partie.
      \end{enumerate}
    \end{itemize}

  \end{enumerate}

\item Méthodes : 

  \begin{enumerate}
  \item terminer : Termine une partie en cours en stoquant la date de fin (ne fait rien si la partie est déjà terminée).
  \item getTemps : Retourne le nombre de seconde qu'a duré la partie (entre la date de début et la date de fin)
    \begin{itemize}
    \item Retour : le nombre de seconde qu'a duré la partie. Si la partie est encore en cours, retourne le nombre de secondes écoulées depuis le début.
      \begin{itemize}
      \item Type de retour: int
      \end{itemize}
    \end{itemize}
  \end{enumerate}
  
\end{itemize}

\section{Classe PartieDAO}

La classe d'accès aux données de la Partie

\begin{itemize}
\item Méthodes : 

  \begin{enumerate}
  \item lire : instancie une Partie à partir des informations provenant de la source de données.
    \begin{itemize}
    \item Paramètres : 
      \begin{enumerate}
      \item identifiant (Object) : L'ID de la Partie
      \end{enumerate}
    \item Retour : Le Partie provenant de la source de données ou null s'il n'a pas été trouvé.
      \begin{itemize}
      \item Type de retour: Partie
      \end{itemize}
    \item Assertions : 
      \begin{enumerate}
      \item identifiant est un Integer.
      \end{enumerate}
    \item Lance :
      \begin{enumerate}
      \item DAOException : si un problème survient lors de la lecture.
      \end{enumerate}
    \end{itemize}

  \item créer : ajoute un nouvelle Partie dans la source de données.
    \begin{itemize}
    \item Paramètres : 
      \begin{enumerate}
      \item unePartie (Partie) : Le Partie à ajouter
      \end{enumerate}
    \item Retour : Le Partie ajoutée à la source de données telle qu'elle peut avoir été modifiée lors de l'ajout.
      \begin{itemize}
      \item Type de retour: Partie
      \end{itemize}
    \item Assertions : 
      \begin{enumerate}
      \item unePartie est non null
      \end{enumerate}
    \item Lance :
      \begin{enumerate}
      \item DAOException : si un problème survient lors de la création.
      \end{enumerate}
    \end{itemize}

  \item modifier : modifie une partie dans la source de données.
    \begin{itemize}
    \item Paramètres : 
      \begin{enumerate}
      \item unePartie (Partie) : Le Partie telle qu'elle doit devenir dans la source de données.
      \end{enumerate}
    \item Retour : Le Partie après modification dans la source de données.
      \begin{itemize}
      \item Type de retour: Partie
      \end{itemize}
    \item Assertions : 
      \begin{enumerate}
      \item unePartie est non null
      \end{enumerate}
    \item Lance :
      \begin{enumerate}
      \item DAOException : si un problème survient lors de la modification.
      \end{enumerate}
    \end{itemize}
    
  \item supprimer : supprime une partie de la source de données.
    \begin{itemize}
    \item Paramètres : 
      \begin{enumerate}
      \item unePartie (Partie) : La Partie à supprimer
      \end{enumerate}
    \item Assertions : 
      \begin{enumerate}
      \item unePartie est non null
      \end{enumerate}
    \item Lance :
      \begin{enumerate}
      \item DAOException : si un problème survient lors de la suppression.
      \end{enumerate}
    \end{itemize}

  \item trouverTout : Recherche toutes les parties à partir de la source de données
    \begin{itemize}
    \item Retour : Une liste de Parties représentant toutes les parties trouvées dans la source de données
      \begin{itemize}
      \item Type de retour: ArrayList<Partie>
      \end{itemize}
    \item Lance :
      \begin{enumerate}
      \item DAOException : si un problème survient lors de la recherche.
      \end{enumerate}
    \end{itemize}

  \item trouverParDifficulté : Recherche toutes les parties d'un certain niveau de difficulté dans la source de données.
    \begin{itemize}
    \item Paramètres : 
      \begin{enumerate}
      \item difficulté (Niveau) : Le niveau de difficulté des parties recherchées
      \end{enumerate}
    \item Retour : Une liste de Parties représentant toutes les parties trouvés dans la source de données
      \begin{itemize}
      \item Type de retour: ArrayList<Partie>
      \end{itemize}
    \item Lance :
      \begin{enumerate}
      \item DAOException : si un problème survient lors de la recherche.
      \end{enumerate}
    \end{itemize}

  \item trouverParDate : Recherche toutes les parties ayant débuté à une certaine date dans la source de données.
    \begin{itemize}
    \item Paramètres : 
      \begin{enumerate}
      \item date (Date) : La date de début des parties recherchées
      \end{enumerate}
    \item Retour : Une liste de Parties représentant toutes les parties trouvées dans la source de données
      \begin{itemize}
      \item Type de retour: ArrayList<Partie>
      \end{itemize}
    \item Lance :
      \begin{enumerate}
      \item DAOException : si un problème survient lors de la recherche.
      \end{enumerate}
    \end{itemize}
    
  \end{enumerate}

\end{itemize}
