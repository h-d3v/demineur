\section {Classe Joueur}

Un joueur de démineur. Il est identifié de façon unique par son pseudonyme.

Le niveau d'un joueur est donné par le nombre de points accumulé en gagnant des parties (indépendamment du nombre de parties total jouées). Selon le niveau de difficulté d'une partie, le joueur reçoit un certain nombre de points de la façon suivante :

\begin{enumerate}
\item Facile : 1 point
\item Moyen : 4 points
\item Difficile : 10 points
\end{enumerate}

\begin{itemize}

\item Propriétés : 
  \begin{enumerate}
  \item nom : Le nom complet du joueur 
    \begin{itemize}
    \item Type : String
    \item Accès : lecture seule
    \end{itemize}
  \item pseudonyme : Le pseudonyme du joueur
    \begin{itemize}
    \item Type : String
    \item Accès : lecture seule
    \end{itemize}
  \item niveau : un niveau numérique représentant la qualité du joueur.
    \begin{itemize}
    \item Type : int
    \item Accès : lecture seule
    \end{itemize}
  \item parties : La liste des parties jouées par le joueur
    \begin{itemize}
    \item Type : ArrayList<Partie>
    \item Accès : lecture seule
    \end{itemize}
  \end{enumerate}

\item Constructeurs : 

  \begin{enumerate}
  \item Joueur : Constructeur complet du joueur. Utile pour construire un joueur en précisant toutes ses caractéristiques.
    \begin{itemize}
    \item Paramètres : 
      \begin{enumerate}
      \item unNom (String) : Le nom du Joueur
      \item unPseudonyme (String) : Le pseudonyme du Joueur
      \item unNiveau (int) : Le niveau du Joueur
      \item desParties (ArrayList<Partie>) : La liste des parties du Joueur
      \end{enumerate}
    \item Assertions : 
      \begin{enumerate}
      \item unNom ne peut pas être null ou une chaîne vide.
      \item unPseudonyme ne peut pas être null ou une chaîne vide.
      \item unNom ne peut pas être < 0.
      \item desParties ne peut pas être null.
      \end{enumerate}
    \end{itemize}
  \item Joueur : Constructeur avec valeurs par défaut. Utile pour un joueur qui vient de s'inscrire. Le Joueur est créé avec son nom et son pseudonyme fournis, un niveau à 0 et une liste de parties vide.
    \begin{itemize}
    \item Paramètres : 
      \begin{enumerate}
      \item unNom (String) : Le nom du Joueur
      \item unPseudonyme (String) : Le pseudonyme du Joueur
      \end{enumerate}
    \item Assertions : 
      \begin{enumerate}
      \item unNom ne peut pas être null ou une chaîne vide.
      \item unPseudonyme ne peut pas être null ou une chaîne vide.
      \end{enumerate}
    \end{itemize}

  \end{enumerate}

\item Méthodes : 

  \begin{enumerate}
  \item ajouterPartie : ajoute une partie à la liste des parties 
    \begin{itemize}
    \item Paramètres : 
      \begin{enumerate}
      \item unePartie (Partie) : La partie à ajouter
      \end{enumerate}
    \item Assertions : 
      \begin{enumerate}
      \item La partie n'est pas nulle
      \end{enumerate}
    \end{itemize}
  \end{enumerate}
  \end{itemize}

\section {Classe JoueurDAO}

La classe d'accès aux données du Joueur.

\begin{itemize}
\item Hérite de : DAO<Joueur>

\item Méthodes : 

  \begin{enumerate}
  \item lire : instancie un Joueur à partir des informations provenant de la source de données.
    \begin{itemize}
    \item Paramètres : 
      \begin{enumerate}
      \item identifiant (Object) : Le pseudonyme du Joueur
      \end{enumerate}
    \item Retour : Le Joueur provenant de la source de données ou null s'il n'a pas été trouvé.
      \begin{itemize}
      \item Type de retour: Joueur
      \end{itemize}
    \item Assertions : 
      \begin{enumerate}
      \item identifiant est une chaîne de caractère non nulle et non vide.
      \end{enumerate}
    \item Lance :
      \begin{enumerate}
      \item DAOException : si un problème survient lors de la lecture.
      \end{enumerate}
    \end{itemize}

  \item créer : ajoute un nouveau Joueur dans la source de données.
    \begin{itemize}
    \item Paramètres : 
      \begin{enumerate}
      \item unJoueur (Joueur) : Le Joueur à ajouter
      \end{enumerate}
    \item Retour : Le Joueur ajouté à la source de données tel qu'il peut avoir été modifié lors de l'ajout.
      \begin{itemize}
      \item Type de retour: Joueur
      \end{itemize}
    \item Assertions : 
      \begin{enumerate}
      \item unJoueur est non null
      \end{enumerate}
    \item Lance :
      \begin{enumerate}
      \item DAOException : si un problème survient lors de la création.
      \end{enumerate}
    \end{itemize}

  \item modifier : modifie un Joueur dans la source de données.
    \begin{itemize}
    \item Paramètres : 
      \begin{enumerate}
      \item unJoueur (Joueur) : Le Joueur tel qu'il doit devenir dans la source de données.
      \end{enumerate}
    \item Retour : Le Joueur après modification dans la source de données.
      \begin{itemize}
      \item Type de retour: Joueur
      \end{itemize}
    \item Assertions : 
      \begin{enumerate}
      \item unJoueur est non null
      \end{enumerate}
    \item Lance :
      \begin{enumerate}
      \item DAOException : si un problème survient lors de la modification.
      \end{enumerate}
    \end{itemize}
    
  \item supprimer : supprime un Joueur de la source de données.
    \begin{itemize}
    \item Paramètres : 
      \begin{enumerate}
      \item unJoueur (Joueur) : Le Joueur à supprimer
      \end{enumerate}
    \item Assertions : 
      \begin{enumerate}
      \item unJoueur est non null
      \end{enumerate}
    \item Lance :
      \begin{enumerate}
      \item DAOException : si un problème survient lors de la modification.
      \end{enumerate}
    \end{itemize}

  \item trouverTout : Recherche tous les joueurs à partir de la source de données
    \begin{itemize}
    \item Retour : Une liste de Joueurs représentant tous les joueurs trouvés dans la source de données
      \begin{itemize}
      \item Type de retour: ArrayList<Joueur>
      \end{itemize}
    \item Lance :
      \begin{enumerate}
      \item DAOException : si un problème survient lors de la modification.
      \end{enumerate}
    \end{itemize}
    
  \item trouverParNiveau : Recherche tous les joueurs d'un certain niveau dans la source de données.
    \begin{itemize}
    \item Paramètres : 
      \begin{enumerate}
      \item niveau (int) : Le niveau des joueurs recherchés
      \end{enumerate}
    \item Retour : Une liste de Joueurs représentant tous les joueurs trouvés dans la source de données
      \begin{itemize}
      \item Type de retour: ArrayList<Joueur>
      \end{itemize}
    \item Assertions : 
      \begin{enumerate}
      \item niveau est >= 0.
      \end{enumerate}
    \item Lance :
      \begin{enumerate}
      \item DAOException : si un problème survient lors de la modification.
      \end{enumerate}
    \end{itemize}
  \end{enumerate}
\end{itemize}
